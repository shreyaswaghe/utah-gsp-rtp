\documentclass[11pt, letterpaper]{scrartcl}
\usepackage[T1]{fontenc}
\usepackage{geometry}
\usepackage{amsmath, amssymb, bm}
\usepackage{physics, siunitx}
\usepackage[shortlabels]{enumitem}
\usepackage{amsthm}
\usepackage{hyperref}

\geometry{margin=2cm}
\sisetup{separate-uncertainty=true, exponent-product=\cdot, range-units=single}
\hypersetup{colorlinks=true, linkcolor=blue, urlcolor=cyan}
\setlength{\parindent}{0pt}
\setlength{\parskip}{7pt}

\newcommand{\diff}{\mathop{}\!\mathrm{d}}
\newcommand{\TODO}[1]{{\textbf{TODO:} {\color{red} #1}}}

\input{~/.config/macros.tex}

\title{Cover times for a Run-and-Tumble particle on a bounded interval}
\subtitle{University of Utah Graduate School Preview Project}
\author{}
\date{}

\begin{document}
\maketitle


\section{Introduction}

Consider a particle exhibiting dynamics in one dimension, governed by the Langevin equation
\begin{equation}
    \label{eq:langevin}
    \dv{x}{t} = v \cdot \sigma(t)
\end{equation}
where $ v $ is a constant speed, and the polarity, $ \sigma(t) \in \{ \pm 1 \} $, determines the random, noisy evolution of the particle.
The polarity switches between its two values $ \pm 1 $ as a Poisson process with rate $ \gamma $, i.e., the times between two consecutive switches is a Random Variable distributed as $ \text{Exp}(\gamma) $.
The setting we consider is of \textbf{Stochastic Resetting and Velocity Randomization}, wherein the particle \emph{resets} to a point (which may or may not coincide with its original position), also as a Poisson process with rate $r$.

This system, in its multidimensional equivalent, is reminiscent of a number of common problems across scientific disciplines, from stochastic search algorithms in computer science to the behavior of animals foraging for food in ecology (citation).

Two important terms we will encounter in this short document are the \textbf{First Passage Time} (FPT) and the \textbf{Minimum Cover Time} (MCT), which are intimately related to one another:
\begin{enumerate}

    \item The FPT of a particle which \textit{spawns} or \textit{originates} at the point $x_0 > 0$ is the minimum time at which it hits a target $T$, assumed to be the origin without loss of generality.
    \begin{equation}
        \tau_{\text{FPT}} = \inf \{ t |  \} 
    \end{equation}
\end{enumerate}

\end{document}

